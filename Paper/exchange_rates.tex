% This is "sig-alternate.tex" V2.1 April 2013
% This file should be compiled with V2.5 of "sig-alternate.cls" May 2012
%
% This example file demonstrates the use of the 'sig-alternate.cls'
% V2.5 LaTeX2e document class file. It is for those submitting
% articles to ACM Conference Proceedings WHO DO NOT WISH TO
% STRICTLY ADHERE TO THE SIGS (PUBS-BOARD-ENDORSED) STYLE.
% The 'sig-alternate.cls' file will produce a similar-looking,
% albeit, 'tighter' paper resulting in, invariably, fewer pages.
%
% ----------------------------------------------------------------------------------------------------------------
% This .tex file (and associated .cls V2.5) produces:
%       1) The Permission Statement
%       2) The Conference (location) Info information
%       3) The Copyright Line with ACM data
%       4) NO page numbers
%
% as against the acm_proc_article-sp.cls file which
% DOES NOT produce 1) thru' 3) above.
%
% Using 'sig-alternate.cls' you have control, however, from within
% the source .tex file, over both the CopyrightYear
% (defaulted to 200X) and the ACM Copyright Data
% (defaulted to X-XXXXX-XX-X/XX/XX).
% e.g.
% \CopyrightYear{2007} will cause 2007 to appear in the copyright line.
% \crdata{0-12345-67-8/90/12} will cause 0-12345-67-8/90/12 to appear in the copyright line.
%
% ---------------------------------------------------------------------------------------------------------------
% This .tex source is an example which *does* use
% the .bib file (from which the .bbl file % is produced).
% REMEMBER HOWEVER: After having produced the .bbl file,
% and prior to final submission, you *NEED* to 'insert'
% your .bbl file into your source .tex file so as to provide
% ONE 'self-contained' source file.
%
% ================= IF YOU HAVE QUESTIONS =======================
% Questions regarding the SIGS styles, SIGS policies and
% procedures, Conferences etc. should be sent to
% Adrienne Griscti (griscti@acm.org)
%
% Technical questions _only_ to
% Gerald Murray (murray@hq.acm.org)
% ===============================================================
%
% For tracking purposes - this is V2.0 - May 2012

\documentclass{sig-alternate-05-2015}
\usepackage{amsmath}
%\usepackage{algorithm}
\usepackage{algorithm2e}
\usepackage[noend]{algpseudocode}

\begin{document}


% Copyright
%\setcopyright{acmcopyright}
%\setcopyright{acmlicensed}
%\setcopyright{rightsretained}
%\setcopyright{usgov}
%\setcopyright{usgovmixed}
%\setcopyright{cagov}
%\setcopyright{cagovmixed}


\title{{Predicting Exchange Rates}
}
\numberofauthors{2} 
\author{
\alignauthor
Eduardo Lorie\\
       \affaddr{Georgia Institute of Technology}\\
       \email{edlorie@gatech.edu}
\alignauthor 
Matthew Robinson \\
       \affaddr{Georgia Institute of Technology}\\
       \email{mrobinson72@gatech.edu}
}

\date{31 January 2016}

\maketitle

\section{Introduction}

\section{Theoretical Framework}
Research on equilibrium in foreign exchange markets is a well developed component of classical economic theory. Classical models rest on two fundamental ideas. The first is that foreign exchange markets achieve equilibrium when the rate of return on deposits is the same across all currencies. The idea that investors will be indifferent between bank deposits denominated in different currencies is known as interest rate parity. The second is that the price of goods will be the same when valued in different currencies. The notion that a basket of goods should cost the same in all currencies is known as purchasing power parity.
\par{} 

\section{Data}

\section{Results}

\section{Conclusion}
     
% hangref environment
\newenvironment{hangref}{\begin{list}{}{\setlength{\itemsep}{0pt}
\setlength{\parsep}{0pt}\setlength{\rightmargin}{0pt}
\setlength{\leftmargin}{+\parindent}
\setlength{\itemindent}{-\parindent}}}{\end{list}}
\section*{REFERENCES}
\begin{hangref}

\item Frankel, J., and A. Rose. 1996.
``A panel project on purchasing power parity: mean reversion within and between countries.''
{\it Journal of International Economics} 40: 209-224.

\item Krugman, P., M. Obstfeld and M.Melitz.  2014.
{\it International Economics: Theory and Policy}. 10th ed.
Upper Saddle River, New Jersey: Pearson Education.

\item Juselius, K. 1995.
``Do purchasing power parity and uncovered interest rate parity hold in the long run? An example of likelihood inference in a multivariate time-series model.''.
{\it Journal of Econometrics} 69: 211-240.

\item Obstfeld, M., and A. Taylor. 2003.
``Globalization and capital markets.'' 
{\it Globalization in historical perspective.}
University of Chicago Press: 121-188.

\item Wooldridge, J. 2009.
{\it Introductory Econometrics}. 4th Ed.
Mason, Ohio: South-Western Cengage Learning.





\end{hangref}

\clearpage





\end{document}
