% This is "sig-alternate.tex" V2.1 April 2013
% This file should be compiled with V2.5 of "sig-alternate.cls" May 2012
%
% This example file demonstrates the use of the 'sig-alternate.cls'
% V2.5 LaTeX2e document class file. It is for those submitting
% articles to ACM Conference Proceedings WHO DO NOT WISH TO
% STRICTLY ADHERE TO THE SIGS (PUBS-BOARD-ENDORSED) STYLE.
% The 'sig-alternate.cls' file will produce a similar-looking,
% albeit, 'tighter' paper resulting in, invariably, fewer pages.
%
% ----------------------------------------------------------------------------------------------------------------
% This .tex file (and associated .cls V2.5) produces:
%       1) The Permission Statement
%       2) The Conference (location) Info information
%       3) The Copyright Line with ACM data
%       4) NO page numbers
%
% as against the acm_proc_article-sp.cls file which
% DOES NOT produce 1) thru' 3) above.
%
% Using 'sig-alternate.cls' you have control, however, from within
% the source .tex file, over both the CopyrightYear
% (defaulted to 200X) and the ACM Copyright Data
% (defaulted to X-XXXXX-XX-X/XX/XX).
% e.g.
% \CopyrightYear{2007} will cause 2007 to appear in the copyright line.
% \crdata{0-12345-67-8/90/12} will cause 0-12345-67-8/90/12 to appear in the copyright line.
%
% ---------------------------------------------------------------------------------------------------------------
% This .tex source is an example which *does* use
% the .bib file (from which the .bbl file % is produced).
% REMEMBER HOWEVER: After having produced the .bbl file,
% and prior to final submission, you *NEED* to 'insert'
% your .bbl file into your source .tex file so as to provide
% ONE 'self-contained' source file.
%
% ================= IF YOU HAVE QUESTIONS =======================
% Questions regarding the SIGS styles, SIGS policies and
% procedures, Conferences etc. should be sent to
% Adrienne Griscti (griscti@acm.org)
%
% Technical questions _only_ to
% Gerald Murray (murray@hq.acm.org)
% ===============================================================
%
% For tracking purposes - this is V2.0 - May 2012

\documentclass{sig-alternate-05-2015}
\usepackage{amsmath}
%\usepackage{algorithm}
\usepackage{algorithm2e}
\usepackage[noend]{algpseudocode}
\usepackage{eurosym}

\begin{document}


% Copyright
%\setcopyright{acmcopyright}
%\setcopyright{acmlicensed}
%\setcopyright{rightsretained}
%\setcopyright{usgov}
%\setcopyright{usgovmixed}
%\setcopyright{cagov}
%\setcopyright{cagovmixed}


\title{{Using Machine Learning to Set Exchange Rates for Medium Term Contracts}
}
\numberofauthors{2} 
\author{
\alignauthor
Eduardo Lorie\\
       \affaddr{Georgia Institute of Technology}\\
       \email{edlorie@gatech.edu}
\alignauthor 
Matthew Robinson \\
       \affaddr{Georgia Institute of Technology}\\
       \email{mrobinson72@gatech.edu}
}

\date{20 February 2016}

\maketitle

\section{Introduction}
International business involves considerable risk. This is especially true for manufacturers, who often make production decisions well in advance of delivery dates. Consider the position of an airline that is preparing to expand its fleet. A typical contract for this purpose might involve an order for a passenger jet that will not be completed for a year or more, with payment due upon delivery of the jet. If the buyer is American and the manufacturer is also American, this is not a problem. All of the manufacturers production costs will be valued in US dollars, and the customer will pay in US dollars.
\par{} In contrast, consider the situation if the buyer is Canadian and the manufacturer is American. Since the manufacturer is American, the Canadian firm must pay for the jet in US dollars. Suppose the value of the US dollar appreciates 10\% relative to the Canadian dollar while the American firm is filling the order. Then, when the Canadian firm pays for the jet, it is 10\% more expensive in terms of Canadian dollars than when it ordered it. If the movement were in the opposite direction, the manufacturer would receive 10\% fewer US dollars for the jet than it expected when it was ordered. Clearly, both firms would be interested in predicting adverse currency movements in advance, and may be interested in writing expected future currency valuations into contracts. 
\par{} This paper seeks to resolve this dilemma by estimating a model that predicts the percent change in the exchange rate after one year, using only information that is currently available. The remainder of this section will provide an explicit problem definition, as well as an overview of prevailing economic theory and past statistical research on exchange rates. In the following section, we explain the data set and data sources. Next, we develop several models for predicting exchange rates and compare their effectiveness. Finally, we conclude by suggesting the most effective means to predict exchange rates and possible directions for future research.

\subsection{Problem Definition}


\subsection{Related Work}
Research on equilibrium in foreign exchange markets is a well developed component of classical economic theory. Classical models rest on two fundamental ideas. The first is that foreign exchange markets achieve equilibrium when the rate of return on deposits is the same across all currencies. The idea that investors will be indifferent between bank deposits denominated in different currencies is known as interest rate parity. The second is that the price of goods will be the same when valued in different currencies. The notion that a basket of goods should cost the same in all currencies is known as purchasing power parity.
\par{} Both of these concepts rely on the same basic premise, which is that price differentials create arbitrage opportunities. Interest rate and purchasing power parity hold that price differentials are self-correcting because, as investors move to take advantage of arbitrage opportunities, they push the market back toward equilibrium. To demonstrate this idea, suppose that a laptop costs \$500 in the United States and the Euro equivalent of \$550 in Germany. Someone in Germany could take advantage of this fact by buying cheap laptops in the US and selling them in Germany. This would create more demand for dollars, which are required to buy the laptops in the US, and push up US price levels. This trend would continue until price levels are high enough that laptops cost the same in the US as in Germany. When laptop prices are the same in both countries, the market is in equilibrium since there is no longer an opportunity for arbitrage.
\par{} The following equations summarize the relationship between exchange rates and interest rates under interest rate parity and the relationship between exchange rates and price levels under purchasing power parity, all else held equal. In these equations, $e_{t}$ represents the exchange rate at time $t$ in terms of the foreign currency, $i$ is the real interest rate and $\pi$ is the inflation rate. 
\begin{equation}
e_{t+1} = e_{t} \left( \frac{1+i_{d}}{1+i_{f}} \right)
\end{equation}
\begin{equation}
\frac{e_{t+1}-e_{t}}{e_{t}} = \frac{1+\pi_{d}}{1+\pi{f}} - 1
\end{equation}
\par{} Obstfeld and Taylor show that arbitrage opportunities between foreign and domestic assets are near zero under floating currency regimes, which suggests that exchange rates behave as predicted by interest rate parity. Likewise, Frankel and Rose provide evidence that exchange rates converge to levels predicted by purchasing power parity in the long run. The relevant time-frame for the \emph{long run} in this context is about four years. Under shorter time horizons, purchasing power parity performs considerably less well. Likewise, difficulty in predicting the relative performance of financial assets in different countries makes it difficult to predict interest rates using interest rate parity. In fact, Meese and Rogoff famously demonstrated that a random walk outperforms structural models for the exchange rate over a one to twelve month window. For this reason, the random walk is currently used as a benchmark for assessing the quality of models for predicting exchange rates.
\par{} [overview of ML literature on exchange rates here ...]

\section{Data}
This paper restricts its attention to changes in the exchange rate between the United States and its five biggest trading partners, excluding China. These trading partners are Europe, Canada, Mexico, Japan and South Korea. Europe in this context refers to the 19 countries that belong to the eurozone. Since they have a single monetary policy, they will be treated as a single unit for the purposes of this paper. China is excluded because it fixes its exchange rate. Because of this, its exchange rate is determined by government policy rather than economic variables.
\par{} The data used in this paper was collected entirely by central banks and compiled by Quandl, an economic data repository. The primary variables of interest in the data set are the inflation rate, interest rate and the yield on one year government bonds. In this data set, the inflation rate measures the annualized percent change in the consumer price index for each country. The interest rate is the interbank lending rate set by the central bank for the currency in question. For the United States, this is the federal funds rate.  Other variables in the data set include an area's balance of trade, current account, foreign exchange reserves and other factors related to international financial flows. These variables were measured both for the United States and each currency pairing. Each variable is measured monthly, with the exception of GDP growth, current account balance and foreign direct investment. These variables are measured quarterly. In order to align this data with the rest of the data set, values for the intervening two months between each quarter were interpolated using cubic splines.

\section{Results}

\section{Conclusion}
     
% hangref environment
\newenvironment{hangref}{\begin{list}{}{\setlength{\itemsep}{0pt}
\setlength{\parsep}{0pt}\setlength{\rightmargin}{0pt}
\setlength{\leftmargin}{+\parindent}
\setlength{\itemindent}{-\parindent}}}{\end{list}}
\section*{REFERENCES}
\begin{hangref}

\item Frankel, J., and A. Rose. 1996.
``A panel project on purchasing power parity: mean reversion within and between countries.''
{\it Journal of International Economics} 40: 209-224.

\item Krugman, P., M. Obstfeld and M.Melitz.  2014.
{\it International Economics: Theory and Policy}. 10th ed.
Upper Saddle River, New Jersey: Pearson Education.

\item Juselius, K. 1995.
``Do purchasing power parity and uncovered interest rate parity hold in the long run? An example of likelihood inference in a multivariate time-series model.''.
{\it Journal of Econometrics} 69: 211-240.

\item Meese, R., and K. Rogoff. 1983.
``Empirical exchange rate models of the seventies: Do they fit out of sample?.'' 
{\it Journal of International Economics} 14: 3-24.

\item Obstfeld, M., and A. Taylor. 2003.
``Globalization and capital markets.'' 
{\it Globalization in historical perspective}
University of Chicago Press: 121-188.

\item Wooldridge, J. 2009.
{\it Introductory Econometrics}. 4th Ed.
Mason, Ohio: South-Western Cengage Learning.





\end{hangref}

\clearpage





\end{document}
